\subsection{by Lillian Guo}
\textbf{Carroll, S. M., Chen, J. (2005) Does inflation provide natural initial conditions for the universe}

This paper by Carroll and his graduate student Chen \cite{Carroll2005} addresses the problem of initial conditions of inflation, which is a period of accelerated expansion the universe underwent in its very early phase proposed to address problems of the standard Hot Big Bang cosmology (e.g. requiring fine-tuned initial conditions). They address this question through considering the entropy of the universe: currently the universe has not reached maximum entropy (or else there would be no more evolution), but its entropy during inflation would have been much, much lower. Such low entropy state is unlikely to be the result of a random choice, and therefore inflation appears unlikely to have arisen randomly. However, if inflation itself requires fine-tuned initial conditions, it would not satisfactorily resolve issues it was conceived for.\par 

Carroll and Chen resolve this issue through proposing that the natural state for the universe is near-empty (de Sitter) spacetime, and instead of starting from a random field, inflation started in this near-empty state, which admits spontaneous onset of inflation. \par 

The central concepts in this argument are the Second Law of Thermodynamics: $dS_\mathrm{universe} \geq 0$, and entropy.\par 

Before inflation, the entropy of a patch which grows subsequently to our visible universe is of the order $S_I \propto \text{Area of horizon} \sim (H_I^{-1}/L_P)^2 \sim 10^{12}$ where $H_I^{-1}$ is the Hubble radius (for speed of light $c=1$, the horizon is roughly $1/H$ for $H$ the Hubble parameter) and $L_P$ the Planck length\footnote{Details can be found at http://www.hartmanhep.net/topics2015/7-deSitterv2.pdf}.\par 

Shortly after this, the entropy was dominated by radiation, which we may calculate with \cite{hirata} 
\begin{equation*}
s=\int_{0} \frac{d \rho}{T}=\int_{0}^{T} \frac{1}{T} \frac{d \rho}{d T} d T
\end{equation*}
where we now consider the entropy and energy densities $s$ and $\rho$. The energy density is found by integrating over phase space
\begin{equation*}
\rho=\sum_{i} \frac{g_{i}}{2 \pi^{2}} \int_{0}^{\infty} p^{2} \sqrt{p^{2}+m_{i}^{2}} f_{i}(p) d p=\sum_{i} \frac{g_{i}}{2 \pi^{2}} \int_{0}^{\infty} \frac{p^{2} \sqrt{p^{2}+m_{i}^{2}} d p}{e ^ {\sqrt{p^{2}+m_{i}^{2}} / T} \mp 1}
\end{equation*}
where we have used the Bose-Einstein or Fermi-Dirac distributions for $f_{i}(p)$, and $g_i$ is the degrees of freedom for species $i$. Then taking the limit $T \gg m$, we can evaluate $\rho$ as in Question 1 of this assignment, and get
\begin{equation*}
\rho=\frac{\pi^{2} g T^{4}}{30}
\end{equation*}
when we sum over $g_i$ for the relativistic species $i$. Then
\begin{equation*}
s=\int_{0}^{T} \frac{1}{T} \frac{d}{d T}\left(g \frac{\pi^{2} T^{4}}{30}\right) d T \approx g \int_{0}^{T} \frac{1}{T} \frac{d}{d T} \frac{\pi^{2} T^{4}}{30} d T=\frac{2 \pi^{2} g T^{3}}{15},
\end{equation*}
and $s$ is estimated from the temperature of the Cosmic Microwave Background. The total entropy is then on the order of $S \sim 10^{88}$. It is also interesting to note that since $s\propto T^3$, and $T\propto 1/a$ for $a$ the scale factor describing the amount of universe expansion, the total entropy $S = sV \propto a^{-3}a^{3}$ in a comoving volume is unchanged during radiation domination.\cite{Lineweaver}\par 

The entropy budget of today is dominated by black holes. The dimensionless Bekenstein-Hawking entropy of a black hole is proportional to its horizon area \cite{Egan2010}
\begin{equation*}
S_{BH} = \frac{c^3 A}{4G\hbar} \sim 10^{77}\left(\frac{M_{BH}}{M_\odot}\right)^2.
\end{equation*}
There are billions of galaxies in the observable universe with million-solar-mass black holes at their centres, so the current entropy in black holes is at least $S_{BH,tot} \sim 10^{99}$.\cite{Carroll2005}\par 

We thus see that the entropy today is much higher than before inflation. The authors propose to explain the initial condition problem by stating that the natural state of the universe is de Sitter since this allows greater volume and thus higher entropy (so the universe has a large volume with its constituents scattered far apart), therefore it is natural that before inflation the background spacetime was also de Sitter. Then fluctuations are not random in some measure on phase space of initial conditions, and it is easier for a part of the field to fluctuate into inflationary initial conditions than for a collection to fluctuate to e.g. our universe today \cite{Carroll2004a}. Note also that while the total entropy of de Sitter space is high, the entropy density is small so locally, the inflationary patch has higher entropy than the background.\par 

The question of the naturalness of inflation seems well explained by the authors by noting that de Sitter space is unstable to fluctuations, but their viewpoint also has interesting consequences, such as a universe that can start inflation in any direction of time forwards and backwards from initial conditions, and the possibility of eternal inflation, a concept still under much debate today (\cite{Carroll2004}, \cite{Carroll2004a}).